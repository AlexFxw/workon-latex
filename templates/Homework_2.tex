% --------------------------------------------------------------
% This is all preamble stuff that you don't have to worry about.
% Head down to where it says "Start here"
% --------------------------------------------------------------
 
\documentclass[12pt]{article}
 
\usepackage[margin=1in]{geometry} 
\usepackage{amsmath,amsthm,amssymb}
\usepackage[margin=1in]{geometry} 
\usepackage{amsmath,amsthm,amssymb}
\usepackage[spanish]{babel} %Castellanización
\usepackage[T1]{fontenc} %escribe lo del teclado
\usepackage[utf8]{inputenc} %Reconoce algunos símbolos
\usepackage{lmodern} %optimiza algunas fuentes
\usepackage{graphicx}
\graphicspath{ {images/} }
\usepackage{hyperref} % Uso de links
\usepackage[UTF8]{ctex}
 
\newcommand{\N}{\mathbb{N}}
\newcommand{\Z}{\mathbb{Z}}
 
\newenvironment{theorem}[2][Theorem]{\begin{trivlist}
\item[\hskip \labelsep {\bfseries #1}\hskip \labelsep {\bfseries #2.}]}{\end{trivlist}}
\newenvironment{lemma}[2][Lemma]{\begin{trivlist}
\item[\hskip \labelsep {\bfseries #1}\hskip \labelsep {\bfseries #2.}]}{\end{trivlist}}
\newenvironment{exercise}[2][Exercise]{\begin{trivlist}
\item[\hskip \labelsep {\bfseries #1}\hskip \labelsep {\bfseries #2.}]}{\end{trivlist}}
\newenvironment{problem}[2][Problem]{\begin{trivlist}
\item[\hskip \labelsep {\bfseries #1}\hskip \labelsep {\bfseries #2.}]}{\end{trivlist}}
\newenvironment{question}[2][Question]{\begin{trivlist}
\item[\hskip \labelsep {\bfseries #1}\hskip \labelsep {\bfseries #2.}]}{\end{trivlist}}
\newenvironment{corollary}[2][Corollary]{\begin{trivlist}
\item[\hskip \labelsep {\bfseries #1}\hskip \labelsep {\bfseries #2.}]}{\end{trivlist}}

\newenvironment{solution}{\begin{proof}[Solution]}{\end{proof}}
 
\begin{document}
 
% --------------------------------------------------------------
%                         Start here
% --------------------------------------------------------------
 
\title{Tarea 1 Grafica tu curva}
\author{Carlos Roberto Martínez Hernández\\ %replace with your name
Matemáticas para ingeniería I}

\maketitle
\section{Curvas}
A continuación se presenta la representación gráfica de dos curvas, simultáneamente, cuyas ecuaciones están dadas por:


\
\[\left\{ \begin{array}{rcl}
f_{1}=(cos(t),tan(t))
\\
f_{2}=(tan(t),sin(t)) 
& 
\end{array}
\right. \]


% Y este la incluye con etiqueta y pie de imagen
\begin{figure}[h]
\centering
\includegraphics[scale=0.65]{Images/curva4.png} 
\caption{Gráfica de dos funciones simultáneas}
\label{etiqueta}
\end{figure}

Por último se presenta el código utilizado en scilab, el cuál contiene los comentarios pertinentes para aclarar que hace cada línea de código. \\\\
t= linspace(0,5*\%)pi';\hfill//Especifíca cada cuanto mostrará los puntos\\
x= [cos(t),tan(t)];\hfill//Función 1\\
y= [tan(t),sin(t)];\hfill         //Función 2\\
z= [t,t];\hfill//La variable parametrizada\\
param3d1(x,y,list(z,[5,2]))\hfill//param 3d1 grafica varias funciones a la vez
title("Carlos Roberto Martínez Hernández") \hfill// title, pone título a la imágen\\
legend(["f1";"f2"])\hfill// Aclara a que color corresponde que función

% --------------------------------------------------------------
%     You don't have to mess with anything below this line.
% --------------------------------------------------------------
 
\end{document}